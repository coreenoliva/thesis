\chapter{Introduction}
Machine learning is a large and expanding field of computer science with countless applications and designs. In particular, machine learning techniques can be applied to tackle problems such as pattern recognition and pattern learning. These learning applications can be utilised in the computer vision process known as image segmentation. When trained and implemented properly, the use of machine learning can reduce manual workloads, cost and potentially improve accuracy.
\\[1\baselineskip]
Image segmentation is a critical step in assesing and analysing the pathoanatomical conditions in bone for diagnosis or surgery preparation. This result of this process allows for the contouring and overlaying of mask over the desired object in a medical image for a more simpler and comprehensive analysis. Currently being conducted manually, incorporation of machine learning in this process can mitigate human error and increase speed.

\section{Motivation}
In order to analyse and assess the pathoanatomical conditions in bones from MRIs, high quality segmentation is required. Currently, the most effective method is to do this segmentation manually, pixel by pixel. This is a lengthy and meticulous process currently being done by medical professionals.
\\[1\baselineskip]
There is little to no room for human error as the result needs to be highly accurate in order to provide proper analysis. Results of manual segmentation are generally hard to reproduce and are subjected to the well-being of the professional. If the person segmenting the image is tired or distracted, this could lead to the decrease in accuracy. In order to imrpove this process regarding speed, accuracy and cost, a machine learning algorithm and process is proposed to train a model and produce a segmentation scheme.
 

