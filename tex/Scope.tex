\chapter{Scope}
\label{chpt: scope}
Chapter \ref{chpt: scope} provides an outline on the scope of the thesis and detailed lists of the resources used throughout. 
\section{Topic Definition}
This thesis aims to utilise machine learning and computer vision techniques in order to produce a bone segmentation scheme in MR images.
A model will be devised with the use of the Random Forest algorithm and manually segmented images as a learning reference. 
\\[1\baselineskip]
There are numerous parameters and stages that contribute toward the performance of the model. As such, numerous experiments have been conducted in order to determine the optimal parameters and methods required. This will be done by analysing the contribution each aspect has towards the output of segmentation scheme. This analysis will be conducted with sensitivity, specificity rates and ROC curves to monitor progression. 

\section{Aim}
The following are the aims to be achieved:
\begin{enumerate}
	\item To produce a robust and accurate segmentation scheme for hip joints in MRIs.
	\item To provide a solution that reduces the cost and time of manual segmentation while maintain and/or improving segmentation accuracy.
\end{enumerate}
\section{Resources}
There are numerous resources required to conduct the investigations and research throughout the thesis. The resources are listed below in this section. 
\subsection{Data}
\todo{check this \#}Twenty-nine MR images of a hip joint have been used throughout the project. Each of these images have their corresponding labelled segmentation and mask. Example of the image and corresponding label can be seen in figure \ref{fig:dataex} The labelled image is the manually segmented image and the mask is an indication of where the actual image lies within the boundaries. The usage split of this data will be covered in \todo{refer to method}.

\begin{figure}[H]
    \centering
    \subfloat[MRI of bone]{{\includegraphics[width=5cm]{fig/scope_bone.png} }}
    \qquad
    \subfloat[Label image of MRI]{{\includegraphics[width=5cm]{fig/scope_label.png} }}
    \caption{MRI and corresponding label image.}%
    \label{fig:dataex}
\end{figure}


\subsection{Computing Resources}
Various computing resources were utilised throughout this thesis. All computation was done on ac Acer Aspire E5-571G laptop with 8.00 GB of RAM and running on Windows 10.
\\[1\baselineskip]
Work was stored on a local disk which was backed up by Google Drive. Further back-ups were made on an external hard drive for additional measures. Research, notes and discussions were stored using One Note.
\\[1\baselineskip]
As MRIs are stored as .nii files, specific programs are required in order to view and open these files. SMILI \todo{ref this} was used to view and overlay .nii files as required. 
\subsubsection{MATLAB}
MATLAB was the opted program and language to devise the experimentation and methods in. The following toolboxes were implemented:
\todo[inline]{need to provide references for this}
\begin{enumerate}
	\item Image Processing Toolbox
	\item Statistics Toolbox
\end{enumerate} 
\subsubsection{Algorithms and Libraries}
\begin{enumerate}
	\item  NIfTI processor: Tools for NIfTI and ANALYZE image.
	\item  Machine learning algorithm: Random Forest (Matlab implementation).
	\item  Superpixel: SLICO Superpixels.
	\item  Feature Extraction: Supplied feature calculating scripts. \todo{how to do this properly}
\end{enumerate}

The images supplied were NIfTI formatted. An external Matlab library was required in order to handle the processing of NIfTI files. Tools for NIfTI and  ANALYZE image was selected was it had good documentation and was easy to use for the functionality that was required from it.
\\[1\baselineskip]
Random Forest was the selected learning and prediction algorithm. They are ideal for application with this challenge as they are discriminative and powerful. 
\todo[inline]{FINISH THIS} 

For superpixel patches, the SLICO algorithm was used. 
\todo[inline]{why dis}
Feature extraction was conducted with a supplied scripts and functions.



\subsection{Features}
\label{sect:features}
Feature extraction was conducted to be utilised by the Random Forest. A total of 25 features were extracted as listed in table \ref{table: feature} below. The choice of features extracted remained constant throughout the thesis. The features were selected based on findings from \todo[inline]{ref last year thesis here}. 

\begin{table}[H]
\centering
\caption{List of features calculated.}

\begin{tabular}{|l|l|l|l|}
\hline
\textbf{Intensity} & \textbf{Gradient} & \textbf{GLCM} & \textbf{Entropy}\\
\hline
Mean					 &	$\sum L_1$ 					&	Contrast $L_1$		&	Entropy\\
Median				 &	$\sum L_2$					&	Contrast $L_2$		&\\
Standard Deviation	 &  Mean of $L_1$				&	Correlation $L_1$	&\\
Minimum 				 &	Mean of $L_2$				& 	Correlation $L_2$	&\\
Maximum 				 &  	Standard Deviation of $L_1$ 	&	Energy $L_1$		 	&\\ 
Interquartile		 &  Standard Deviation of $L_2$	&	Energy $L_2$			&\\
				     &  Median 						& 	Homogeneity $L_1$	&\\
				   	 &  Minimum						& 	Homogeneity $L_2$	&\\
				    	 & 	Maximum 						& 						&\\ 
				    	 &	Interquartile of $L_1$		&						&\\
\hline				    	 			   			    	 
\end{tabular}
\label{table: feature}

\end{table}