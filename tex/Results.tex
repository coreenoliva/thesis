\chapter{Results}
\label{chpt: results}
Chapter \ref{chpt: results} results provides an analysis to the results obtained throughout the processes in chapter \ref{chpt: method}. In particular, image segmentations and quantitative analysis  will be provided for the proposed segmentation method. The relationships obtained during parameter selection and process testing results will also be presented. 

\section{Segmentation Method}
The proposed segmentation scheme included a training and testing phase as highlighted from section \ref{sect: segapproach}, through to \ref{sect: teststage}. Where the testing phase produced 9 segmentation results. The performance of the segmentation scheme is analysed and evaluated visually and quantitatively. The final parameters can be referred to in table \ref{table:finalparameters}.

\subsection{Visual Segmentation Results}
The axial view image to segment is seen in figure \ref{fig:img1} and the accompanied labelled image in figure \ref{fig:label1}. Furthermore, the segmentation scheme was performed on the coronal view of the joint and the results can be seen in appendix bla \todo{ref appendix coronal}.

\begin{figure}[H]
    \centering
    \subfloat[Image 1 to segment.]{{\includegraphics[height=9cm]{fig/axialres/065/img.png} }\label{fig:img1}}
    \\
    \subfloat[Manual segmentation of image 1.]{{\includegraphics[height=9cm]{fig/axialres/065/label.png} }\label{fig:label1}}
    \caption{Image 1 and its labelled image (manual segmentation).}%
    \label{fig:image1}
\end{figure}

The best produced segmentation can be seen in \ref{fig:res1} below.

\begin{figure}[H]
\centering
\includegraphics[height=10cm]{fig/axialres/065/res.png}
\caption{Segmentation Result of Image 1.}
\label{fig:res1}
\end{figure}

This segmentation is then over-layed over the MRI slice in \ref{fig:resimg1} while the manual segmentation is along side is in \ref{fig:labelimg1} for comparison. 

\begin{figure}[H]
    \centering
    \subfloat[Result segmentation overlay.]{{\includegraphics[height=9cm]{fig/axialres/065/resimg.png} }\label{fig:resimg1}}
    \\
    \subfloat[Manual segmentation overlay]{{\includegraphics[height=9cm]{fig/axialres/065/labelimg.png} }\label{fig:labelimg1}}
    \caption{Result and manual segmentation overlays.}%
    \label{fig:overlay1}
\end{figure}

A total of 9 segmentation images  were produced which can be referred to in images \ref{fig:overlay2} through to \ref{fig:overlay9}. The segmentation result overlay is shown along with the manual segmentation overlay to set the context as to what the segmentation was aimed to achieve.

\begin{figure}[H]
    \centering
    \subfloat[Result segmentation overlay.]{{\includegraphics[scale=0.6]{fig/axialres/017/resimg.png} }}
    \quad
    \subfloat[Manual segmentation overlay]{{\includegraphics[scale=0.6]{fig/axialres/017/labelimg.png} }}
    \caption{Image 2 - Result and manual segmentation overlays.}%
    \label{fig:overlay2}
\end{figure}

\begin{figure}[H]
    \centering
    \subfloat[Result segmentation overlay.]{{\includegraphics[scale=0.6]{fig/axialres/063/resimg.png} }}
    \quad
    \subfloat[Manual segmentation overlay]{{\includegraphics[scale=0.6]{fig/axialres/063/labelimg.png} }}
    \caption{Image 3 - Result and manual segmentation overlays.}%
    \label{fig:overlay3}
\end{figure}
 
 \begin{figure}[H]
    \centering
    \subfloat[Result segmentation overlay.]{{\includegraphics[scale=0.6]{fig/axialres/064/resimg.png} }}
    \quad
    \subfloat[Manual segmentation overlay]{{\includegraphics[scale=0.6]{fig/axialres/064/labelimg.png} }}
    \caption{Image 4 - Result and manual segmentation overlays.}%
    \label{fig:overlay4}
\end{figure}

\begin{figure}[H]
    \centering
    \subfloat[Result segmentation overlay.]{{\includegraphics[scale=0.6]{fig/axialres/066/resimg.png} }}
    \quad
    \subfloat[Manual segmentation overlay]{{\includegraphics[scale=0.6]{fig/axialres/066/labelimg.png} }}
    \caption{Image 5 - Result and manual segmentation overlays.}%
    \label{fig:overlay5}
\end{figure}

\begin{figure}[H]
    \centering
    \subfloat[Result segmentation overlay.]{{\includegraphics[scale=0.6]{fig/axialres/067/resimg.png} }}
    \quad
    \subfloat[Manual segmentation overlay]{{\includegraphics[scale=0.6]{fig/axialres/067/labelimg.png} }}
    \caption{Image 6 - Result and manual segmentation overlays.}%
    \label{fig:overlay6}
\end{figure}
 
 \begin{figure}[H]
    \centering
    \subfloat[Result segmentation overlay.]{{\includegraphics[scale=0.6]{fig/axialres/068/resimg.png} }}
    \quad
    \subfloat[Manual segmentation overlay]{{\includegraphics[scale=0.6]{fig/axialres/068/labelimg.png} }}
    \caption{Image 7 - Result and manual segmentation overlays.}%
    \label{fig:overlay7}
\end{figure}

\begin{figure}[H]
    \centering
    \subfloat[Result segmentation overlay.]{{\includegraphics[scale=0.6]{fig/axialres/069/resimg.png} }}
    \quad
    \subfloat[Manual segmentation overlay]{{\includegraphics[scale=0.6]{fig/axialres/069/labelimg.png} }}
    \caption{Image 8 - Result and manual segmentation overlays.}%
    \label{fig:overlay8}
\end{figure}

\begin{figure}[H]
    \centering
    \subfloat[Result segmentation overlay.]{{\includegraphics[scale=0.6]{fig/axialres/072/resimg.png} }}
    \quad
    \subfloat[Manual segmentation overlay]{{\includegraphics[scale=0.6]{fig/axialres/072/labelimg.png} }}
    \caption{Image 9 - Result and manual segmentation overlays.}%
    \label{fig:overlay9}
\end{figure}
 
 
\subsection{Quantitative Results}

The performance of the segmentation scheme is defined by the accuracy, sensitivity, specificity and testing time. The performance is evaluated based on the models' ability to conduct patch classification and thus, the resulting image pixel accuracies. The performance of the patch classification can be seen in table \ref{table:patchacc} and the overall image accuracy in table \ref{table:pixelacc}. The time taken to produce each segmentation can be seen in table \ref{table:times}.


\begin{table}[H]
\centering
\caption{Patch based performance results of testing images.}

\begin{tabular}{|l|l|l|l|}
\hline
\textbf{Image}	& \textbf{Accuracy} & \textbf{Sensitvity} & \textbf{Specificity}\\
\hline
\textbf{1}		& 1		& 1		& 1\\ 
\hline
\textbf{2} 		& 1		& 1		& 1	\\
\hline
\textbf{3}		& 1		& 1		& 1	\\
\hline
\textbf{4}		& 1		& 1		& 1	\\
\hline
\textbf{5}		& 1		& 1		& 1	\\
\hline
\textbf{6} 		& 1		& 1		& 1	\\
\hline
\textbf{7}		& 1		& 1		& 1	\\
\hline
\textbf{8}		& 1		& 1		& 1\\
\hline
\textbf{9} 		& 1		& 1		& 1\\
\hline				    	 			   			    	 
\end{tabular}
\label{table:patchacc}
\end{table}


\begin{table}[H]
\centering
\caption{Pixel based performance results of testing images.}

\begin{tabular}{|l|l|l|l|l|}
\hline
\textbf{Image}	& \textbf{Accuracy} & \textbf{Sensitvity} & \textbf{Specificity} & \textbf{Time(s)}\\
\hline
\textbf{1}		& 0.979		& 0.988		& 0.970		& 6.85 \\ 
\hline
\textbf{2} 		& 0.976		& 0.978		& 0.975		& 9.24\\
\hline
\textbf{3}		& 0.977		& 0.993 		& 0.964		& 6.42\\
\hline
\textbf{4}		& 0.969		& 0.982		& 0.957		& 6.67\\
\hline
\textbf{5}		& 0.974		& 0.983		& 0.967		& 6.36\\
\hline
\textbf{6} 		& 0.979		& 0.992		& 0.968		& 6.60\\
\hline
\textbf{7}		& 0.976		& 0.994		& 0.955		& 7.96\\
\hline
\textbf{8}		& 0.974		& 0.979		& 0.970		& 6.10\\
\hline
\textbf{9} 		& 0.975		& 0.978		& 0.955		& 6.38\\
\hline				    	 			   			    	 
\end{tabular}
\label{table:pixelacc}
\end{table}

\begin{table}[H]
\centering
\caption{Testing times of segmentation scheme for testing images.}

\begin{tabular}{|l|l|l|l|l|l|l|l|l|l|}
\hline
\textbf{Image}	& \textbf{1} & \textbf{2} & \textbf{3} & \textbf{4} & \textbf{5} &\textbf{6} &\textbf{7} & \textbf{8} &\textbf{9}\\
\hline
\textbf{Time(s)} & 6.85 & 9.24 & 6.42 & 6.67 & 6.36 & 6.60 & 7.96 & 6.10 & 6.38\\
\hline				    	 			   			    	 
\end{tabular}
\label{table:times}
\end{table}

\section{Parameter Search}
The performance of the segmentation scheme is calculated during the parameter search to determine how the parameter affects and contributes to the final segmentation result. This section will display the results obtained from each parameter test that was described in \ref{sect:paramsearch}. The relationships with accuracy along with ROC curves were used to evaluate each test.As the patch classification performance determines the overall image accuracy, the performance is evaluated by the image accuracy.

\subsection{Random Forest Tree Depth Search}
The tree depth of the Random Forest training was tested by varying the tree depth and recording the performance. This test was run for both regression and classification, and models with 10 and 15 training images. 

\subsubsection{Classification}
\begin{figure}[H]
\centering
\includegraphics[width=\linewidth]{fig/class/tree/treeboth.pdf}
\caption{Tree depth versus accuracy for 10 and 15 training images, classification type.}
\label{class:tree}
\end{figure}

\subsubsection{Regression}

\begin{figure}[H]
\centering
\includegraphics[width=\linewidth]{fig/reg/tree/treeboth.pdf}
\caption{Tree depth versus accuracy for 10 and 15 training images, regression type.}
\label{reg:tree}
\end{figure}

\subsection{Number of Training Superpixels Search}
\label{res:trainpix}
This investigation aimed to find the relationship between the number of training superpixels versus image accuracy and testing time. These relationships were analysed for classification and regression models with 10 and 15 training images. The number of training superpixels were tested against varying numbers of testing superpixels and averaged. 

\subsubsection{Classification}
\begin{figure}[H]
    \centering
    \subfloat[Accuracy vs Training Superpixels.]{{\includegraphics[width=\textwidth]{fig/class/trainpix/trainboth.pdf} }}
    \\
    \subfloat[Time vs Training Superpixels. ]{{\includegraphics[width=\textwidth]{fig/class/trainpix/timeboth.pdf} }}
    \caption{Classification - Relationships with number of training superpixels for 10 and 15 training images.}%
    \label{class:trainpix}
\end{figure}

\subsubsection{Regression}
\begin{figure}[H]
    \centering
    \subfloat[Accuracy vs Training Superpixels.]{{\includegraphics[width=\textwidth]{fig/reg/trainpix/trainboth.pdf} }}
    \\
    \subfloat[Time vs Training Superpixels. ]{{\includegraphics[width=\textwidth]{fig/reg/trainpix/timeboth.pdf} }}
    \caption{Regression - Relationships with number of training superpixels for 10 and 15 training images.}%
    \label{reg:trainpix}
\end{figure}


\subsection{Number of Testing Superpixels Search}
The number of testing superpixels were tested for regression and classification models and both 10 and 15 training images and for classification and regression models. They are tested based on the optimal number of training superpixels found in the previous investigation. This search aims to explore the relationship between the number of testing superpixels and accuracy as well as the resulting affect on testing time. For 10 training images, 500 training superpixels was analysed and 800 training superpixels for 15 training images.

\subsubsection{Classification}
\begin{figure}[H]
    \centering
    \subfloat[Accuracy vs Testing Superpixels.]{{\includegraphics[width=\textwidth]{fig/class/testpix/testboth.pdf} }}
    \\
    \subfloat[Time vs Testing Superpixels. ]{{\includegraphics[width=\textwidth]{fig/class/testpix/timeboth.pdf} }}
    \caption{Classification - Relationships with number of testing superpixels for 10 and 15 training images.}%
    \label{class:testpix}
\end{figure}

\subsubsection{Regression}
\begin{figure}[H]
    \centering
    \subfloat[Accuracy vs Testing Superpixels.]{{\includegraphics[width=\textwidth]{fig/reg/testpix/testboth.pdf} }}
    \\
    \subfloat[Time vs Testing Superpixels. ]{{\includegraphics[width=\textwidth]{fig/reg/testpix/timeboth.pdf} }}
    \caption{Regression - Relationships with number of testing superpixels for 10 and 15 training images.}%
    \label{reg:testpix}
\end{figure}

\subsection{Summary of Parameter Testing Results}
This section aims to summarise the findings and results throughout the parameter testing. Table \ref{table:paramres} aims to summarise the optimal parameter values and their accuracies obtained for both classification and regression models with 10 and 15 training images.


\begin{table}[H]
\centering
\caption{Summary of optimal parameters and accuracies. }

\begin{tabular}{|l|l|l|l|l|l|l|}
\hline
	& \multicolumn{2}{|c|}{\textbf{Random Forest}} & \multicolumn{2}{|c|}{\textbf{No. Training}} & \multicolumn{2}{|c|}{\textbf{No. Testing}}	\\
	&\multicolumn{2}{|c|}{\textbf{Tree Depth}}& 	\multicolumn{2}{|c|}{\textbf{Superpixels}} & \multicolumn{2}{|c|}{\textbf{Superpixels}}\\
\hline
Training Images  & Value & Accuracy &  Value & Accuracy & Value & Accuracy\\	
\hline 
\hline
\multicolumn{7}{|c|}{Classification}\\
\hline
10 & 1000 & 0.820 & 500 & 0.923 & 2500 & 0.963 \\
15 & 1000 & 0.896 & 800 & 0.924 & 2500 & 0.963 \\
\hline
\hline
\multicolumn{7}{|c|}{Regression}\\
\hline
10 & 1000 & 0.896 & 500 & 0.925 & 2500 & 0.964\\
15 & 1000 & 0.896 & 800 & 0.925 & 2500 & 0.964\\
\hline
		   		
\end{tabular}
\label{table:paramres}
\end{table}

\section{Process Testing}
With established training parameters, the post-processing stage is further optimised with more tests. These tests are conducted to determine the levels of resolution that will provide the best accuracies with respect to time and how the morphology processes can affect the output result.

\subsection{Multi-Resolution Levels Testing}
This process looks at investigating the affect of different combinations of testing superpixels on the resulting segmentation. The time taken to create the segmentation is also considered noted. The multi-resolution level testing was conducted for both 10 and 15 training images for further comparison. Table \ref{table:multires} provides a summary on the optimal resolution levels for each training image and their performances.

\begin{table}[H]
\centering
\caption{Summary of performance for the optimal resolution schemes.}

\begin{tabular}{|l|l|l|l|}
\hline
&	\textbf{Resolution} & \textbf{Accuracy} & \textbf{Testing Time}\\
\hline
\textbf{10} & 1500-2500 & 0.968 & 19.21\\
\hline				    	 			
\textbf{15} & 1500-2500 & 0.968 & 19.29\\	
\hline		    	 
\end{tabular}
\label{table:multiresres}
\end{table}

\begin{figure}[H]
    \centering
    \subfloat[Accuracy vs Resolution Scheme.]{{\includegraphics[width=\textwidth]{fig/multires/multiresboth.pdf} }}
    \\
    \subfloat[Time vs Resolution Scheme. ]{{\includegraphics[width=\textwidth]{fig/multires/timeboth.pdf} }}
    \caption{Resolution scheme accuracy and time relationships for 10 and 15 training images.}%
    \label{reg: multires10}
\end{figure}

\subsection{Morphology Testing} 
The test for the best morphology operation encompasses a investigations to not only the different morphological operations but the structural element and its' size. Each size and shape of structural element was tested for each type of operation for the 9 testing images. Within each structural shape per operation, the accuracies are averaged per structural shape size. The performance of each morphological operation are summarised by the best performing structural element size for each shape. The overall accuracy for each morphological operation according to the structural elements can be seen in figure \ref{fig:morphres}. Table \ref{table:morphres} provides a summary on the best performing structural element per morphology operation.

\begin{figure}[H]
\centering
\includegraphics[width=\textwidth]{fig/morphres.pdf}
\caption{Morphology operations accuracies based on structural element shape.}
\label{fig:morphres}
\end{figure}

\begin{table}[H]
\centering
\caption{Summary of performance for morphology operations based on structural element.}

\begin{tabular}{|l|l|l|l|}
\hline
\textbf{Operation} &	\textbf{Shape} & \textbf{Size} & \textbf{Accuracy}\\
\hline
Opening & Disk  & 4  & 0.976\\
\hline				    	 			
Closing & Square & 4  & 0.968\\	
\hline
Dilation & Square & 1 & 0.968\\
\hline
Erosion & Disk & 1 & 0.975\\		    	 
\hline 
\end{tabular}
\label{table:morphres}
\end{table}
