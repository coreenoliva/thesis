\chapter{Background}
\section{MRIs}

\section{MachineLearning}
Machine learning is a method that gives computers the ability to solve solutions by learning from supplied data. Machine learning is getting increasingly important as the amount of data and variety grows. The applications of this method become more feasible as computer power gets more powerful and cheaper and algorithms become more advanced and efficient. 

\subsection{Types of Machine Learning}
There are various types of machine learning, some include:
\begin{itemize}
	\item Supervised learning
	\item Unsupervised learning
	\item Reinforcement learning
\end{itemize}

Given a scenario with two possible outcomes, supervised learning learns by being supplied with examples for either outcome. The algorithm then learns and analyses the patterns found in the data and provides a prediction. Supervised learning will be the main focus of this thesis.
\\[1\baselineskip]
In contrast, in unsupervised learning, the algorithm is not given any targets or answers. Instead, it aims to find patterns and sequences that are present in the data. An example application for this type of learning is recommendation engines. Based on trends in browsing history for example, the model will be able to make a prediction on things that would be if similar interest.
\\[1\baselineskip]
Reinforcement learning learns through trial and error. It's output is dependent on a sequence of actions which maximise the end goal or reward. Applications which incorporate this type of learning are games, where the agent learns depending on the output of the environment.
\subsection{Classification}
\subsection{Regression}
	overfitting and underfitting?
\section{Random Forest}
Random Forest is a machine learning algorithm that can give a classification or regression output. It is compromised of many decision trees - making it a forest.  Random forests are discriminative learners that rely on feature learning in order to conduct its classification or regression. 

\subsection{Decision Trees}
Decision trees are a way of mapping out possible decisions and their outcomes in a hierarchical structure 
		\subsection{Classification}
		\subsection{Regression}
\section{Features}
	\subsection{types of features}
\section{Superpixels}
\section{Morphology}
\section{Statistics and Analysis}